\documentclass[10pt]{article}

\usepackage{fullpage}
\usepackage{times}
\usepackage{listings}
\usepackage{hyperref}
\input{chapel_listing}

\textheight 9.5in

\pagestyle{empty}

\begin{document}
\lstset{language=chapel}

\section*{Chapel Hands-on Exercise: The Mandelbrot Set}

\noindent The Mandelbrot Set is the set of points $c$ in the complex
plane for which the sequence defined by the iteration
$z_{n+1} = z_n^2 + c$ remains bounded.  It can be
proven that points outside of $|c| = 2.0$ will cause $z_n$ to escape
to infinity, so we can focus on the interior of that circle.

\vspace{5pt}
\noindent A given $c$ is likely to be \emph{in} the set if the
magnitude of $z_n$ remains less than 2.0 for a reasonably large number
of iterations $N$.  We can stop iterating early if $z_n$ appears to
diverge before $n$ reaches $N$.

\vspace{5pt}
\noindent Pretty pictures you have seen of the Mandelbrot Set assign
each pixel (representing a given $c$) a color corresponding to the
number of iterations required to exceed the bound.  Those with the
color corresponding to $N$ are likely to be in the set; those with
other colors are probably not.

\vspace{10pt}

\noindent
To compute each pixel in a Mandelbrot set rendering, use the
following two steps:
\begin{enumerate}

\item Convert the pixel's location to a value $c$ in the complex
  plane.  The traditional Mandelbrot set image lies within the space
  whose corners are defined by $(-1.5, -1.0i)$ and $(0.5, 1.0i)$; so
  this conversion is simply a linear mapping from integer column
  and row coordinates to a complex number.

  We have provided a function \chpl{mapImg2CPlane} that takes
  in two arguments: an x value (colunmn) and a y value (row)
  and returns the complex value $c$ to be used in the next step.

\item Compute the iterative process described above to see whether or
  setting $c$ to that particular value causes the sequence $z_n$ to
  diverge, and in how many steps.  As pseudocode, this
  algorithm is:

\begin{verbatim}
  // compute whether or not c is in the mandelbrot set
  // input: c, a complex number
  // output: an image value indicating the number of iterations

  start with complex value z equal to zero
  do N steps (in practice a value like 50 produces a nice image)
    replace z with z-squared plus c
    if the absolute value of z exceeds 2.0 then
      store the current number of steps in the image value
      stop iterating
  store zero in the image value if we never exceed 2.0
\end{verbatim}
\end{enumerate}

\vspace{5pt}
\noindent To get experience with Chapel, try the following steps:
\begin{enumerate}
\item {\bf Try the Basic Framework.} The file \chpl{mandelbrot.chpl}
  provides framework code which simply fills the image array with
  values corresponding to the sum of the row and column coordinates,
  scaled to match the image's color depth. The image itself
  is written out as BMP file using a helper module named MPlot.
  You can control the file format,
  image type, and color depth using configuration constants
  to your program if it uses the MPlot module.
  BMP files can be viewed by most image viewers or web
  browsers.

  Compile \chpl{mandelbrot.chpl} and execute it to create an image you can
  view. Have a look at the resulting image file.  It should be filled with
  a diagonal gradient.  Browse the sources or run with the -{}-help option
  to see available configuration options.

\item {\bf A Serial Variant.} 
  Modify the program to fill the image with pixel values using the
  algorithm described above to draw the Mandelbrot set.

\item {\bf A Data-Parallel Version.} 
  Parallelize your serial Chapel program using data parallel features
  such as forall loops or promoted functions/operators.

\item {\bf A Multi-Locale Data-Parallel Version.} 
  Extend your data parallel version to run across multiple locales
  using distributions (domain maps that target multiple locales).  See
  \$CHPL\_HOME/doc/README.multilocale for a guide to getting started
  with multiple locales if using your own Chapel installation; if
  using a pre-installed version provided for the hands-on session,
  refer to instructor-provided directions on using it.  For a brief
  introduction to distributions beyond what was covered in lecture,
  refer to \$CHPL\_HOME/examples/primers/distributions.chpl.

\item {\bf A Task-Parallel Version.} 
  Explicitly parallelize the original serial Chapel program using
  multiple tasks.  Compare this code to your data-parallel
  implementation in terms of effort to write, read, and maintain.

\item {\bf A Multi-Locale Task-Parallel Version (for the particularly brave).} 
  Use Chapel's locales concepts to extend your task-parallel
  implementation to execute using multiple locales.

\end{enumerate}

%\pagebreak
\bigskip
\bigskip

\section*{Explorations}

\begin{itemize}

\item {\bf Explore load balancing.} When decomposed across a large
  number of tasks/processors, the Mandelbrot set can result in load
  imbalance because contiguous regions of the plane will typically
  require a similar number of iterations, some very large, others very
  small.  What load-balancing techniques can you use with your various
  versions to avoid having some tasks/locales complete long before
  others?  You can either write your own techniques or look at some
  standard advanced iterators on ranges provided in the
  \chpl{AdvancedIters} module (documented in the \emph{Chapel Standard
  Library Documentation} - see
  \url{https://chapel-lang.org/docs/modules/standard/AdvancedIters.html})

\item {\bf Explore different regions} near the boundary of the
  Mandelbrot set: Change the boundaries of the subset of the complex
  plane where the computation is carried out.  If you have coded the
  mapping between image coordinates and coordinates in the complex
  plane as a separate procedure or using constants, this should be
  easy.

\item {\bf Alternate color maps.} In the color images, the solutions
  use only the red plane.  How would you apply a pseudocolor map to
  the output?

\item {\bf Expanded dynamic range.} Most of the values which are not
  in the set escape to infinity very rapidly, so the iteration counts
  and resulting color values are all close to zero.  At low
  resolutions, it might yield a more interesting picture if one
  displayed the log of the iteration count.  Can this be done
  efficiently within the computation routine?

\item {\bf Explore parallel I/O.} By default, applying write[ln]() to
  an array will result in its elements being written out serially.
  Parallelizing the I/O may make the entire program execution time
  more scalable (depending on your OS/file system).  See the
  \emph{IO} and/or the \emph{Regexp} standard module documentation
  for more information on Chapel's I/O capabilities:
  \url{https://chapel-lang.org/docs/modules/standard/Regexp.html}
  and
  \url{https://chapel-lang.org/docs/modules/standard/IO.html}.

\end{itemize}

\end{document}
